%-------------------------
% Resume in Latex
% Author : Ran Cheng
% Adapted from: Indu dwivedi, Sourabh Bajaj
% License : MIT
%------------------------

\documentclass[letterpaper,10pt]{article}

\usepackage{latexsym}
\usepackage[empty]{fullpage}
\usepackage{titlesec}
\usepackage{marvosym}
\usepackage[usenames,dvipsnames]{color}
\usepackage{verbatim}
\usepackage{enumitem}
\usepackage[pdftex, hidelinks]{hyperref}
\usepackage{fancyhdr}
\usepackage[charter]{mathdesign} % Bitstream Charter
% \usepackage{newpxtext,newpxmath} % Palatino
\usepackage{longtable}
\usepackage{graphicx}
\usepackage{array}
\usepackage{multirow}
\usepackage{xcolor}
\usepackage{bibentry}
\pagestyle{fancy}
\fancyhf{} % clear all header and footer fields
\fancyfoot{}
\renewcommand{\headrulewidth}{0pt}
\renewcommand{\footrulewidth}{0pt}

% bibtex for publication
\bibliographystyle{plain}
\nobibliography{resume.bib}

% Adjust margins
\addtolength{\oddsidemargin}{-0.50in}
\addtolength{\evensidemargin}{-0.50in}
\addtolength{\textwidth}{1in}
\addtolength{\topmargin}{-.5in}
\addtolength{\textheight}{1.0in}

% Define colors
\definecolor{linkblue}{RGB}{111, 153, 222}
\definecolor{titleblue}{RGB}{46, 116, 181}
\urlstyle{same}

\raggedbottom
\raggedright
\setlength{\tabcolsep}{0in}

% Sections formatting
\titleformat{\section}{
  \vspace{-6pt}\scshape\raggedright\large
}{}{0em}{}[\color{black}\titlerule \vspace{-5pt}]

%-------------------------
% Custom commands
\newcommand{\resumeItem}[2]{
  \item\small{
    \textbf{#1}{: #2 \vspace{-2pt}}
  }
}

\newcommand{\resumeItemNoBullet}[2]{
  \item[]\small{
    \hspace{-9pt}\textbf{#1}{: #2 \vspace{-6pt}}
  }
}

\newcommand{\resumeSubheading}[4]{
  \vspace{-1pt}\item[]
  \begin{tabular*}{0.98\textwidth}{l@{\extracolsep{\fill}}r}
      \hspace{-10pt}\textbf{#1} & #2 \\
      \hspace{-10pt}\textit{\small#3} & \textit{\small #4} \\
    \end{tabular*}\vspace{-5pt}
}

\newcommand{\resumeSubItem}[2]{\resumeItem{#1}{#2}\vspace{-4pt}}

\renewcommand{\labelitemii}{$\circ$}

\newcommand{\resumeSubHeadingListStart}{\begin{itemize}[leftmargin=*]}
\newcommand{\resumeSubHeadingListEnd}{\end{itemize}}
\newcommand{\resumeItemListStart}{\begin{itemize}}
\newcommand{\resumeItemListEnd}{\end{itemize}\vspace{-5pt}}

% custom commands
\newcommand{\shorterSection}[1]{\vspace{-10pt}\section{#1}}

%-------------------------------------------
%%%%%%  CV STARTS HERE  %%%%%%%%%%%%%%%%%%%%%%%%%%%%


\begin{document}

%----------HEADING-----------------
% you can generate your own qr code here: https://www.the-qrcode-generator.com/
% and convert the svg image you exported to pdf here: https://cloudconvert.com/svg-to-pdf
% then import the graph in the title like this:

\begin{table}[]
\begin{tabular*}{\textwidth}{lc@{\extracolsep{\fill}}r}
\begin{tabular}{l}
\textbf{\huge \textcolor{titleblue}{Ru Ji}} \\
\\
\end{tabular}  &  & \begin{tabular}{@{}rr@{}} \textcolor{titleblue}{\includegraphics[width=0.017\linewidth]{imgs/location(1).pdf}} 43 Aitken Circle, Markham, ON, L3R 7L3 &\\
{\includegraphics[width=0.017\linewidth]{imgs/email(1).pdf}} rubujubi@gmail.com   &\\ {\includegraphics[width=0.017\linewidth]{imgs/phone.pdf}} (+1)2267898362                  &                   \\
               
\end{tabular}  \\ 
\end{tabular*}
\end{table}
\vspace*{-10mm}


%-----------EDUCATION-----------------
\shorterSection{\textcolor{titleblue}{Education}}
  \resumeSubHeadingListStart
    \resumeSubheading
      {University of Waterloo, Computer Science}{Waterloo, Canada} {Master of Mathematics (thesis-based)}{Jan 2022 - Jan 2024}{
    %   \resumeItemNoBullet{Thesis}{Guided Robust Visual Navigation with Deep Learning}
      \resumeItemNoBullet{GPA}{89/100}
      \resumeItemNoBullet{Relevant Courses}{Computer Security and Privacy, Advanced Topics in Computer Systems, Advanced Topics in Language Design and Implementation, Advanced Research Topics in Multicore Programming}
    
    \resumeItemNoBullet{Thesis}{Finding False Assurance in Formal Verification of Software Systems, \href{https://uwspace.uwaterloo.ca/bitstream/handle/10012/20231/Ji_Ru.pdf?sequence=3}{\textcolor{linkblue}{link}}}
    \resumeItemNoBullet{Thesis Abstract}{Formal verification plays a crucial role in enhancing the reliability of computing systems
by mathematically checking the correctness of a program. Although recent years have
witnessed lots of research and applications that optimize the formal verification process,
the issue of false assurance persists in certain stages of the formal verification pipeline.
The false assurance problem is critical as it can easily undermine months if not years of
verification efforts.
In this thesis, we first generalized the formal verification process. We then identified
and analyzed specific stages susceptible to false assurance. Subsequently, a systematization
of knowledge pertaining to the false assurance issues observed at these stages is provided,
accompanied by a discussion on the existing defense mechanisms that are currently available.
Specifically, we focused on the problem of formal specification incompleteness. We
presented Fast in this thesis, which is short for Fuzzing-Assisted Specification Testing.
Fast examines the spec for incompleteness issues in an automated way: it first locates
spec gaps via mutation testing, i.e., by checking whether a code variant conforms to
the original spec. If so, Fast further leverages the test suites to infer whether the gap
is introduced by intention or by mistake. Depending on the codebase size, Fast may
choose to generate code variants in either an enumerative or evolutionary way. Fast is
applied to two open-source codebases that feature formal verification and helps to confirm
13 and 21 blind spots in their spec respectively. This highlights the prevalence of spec
incompleteness in real-world applications.}
    \resumeItemNoBullet{Supervisor}{Meng Xu, Assistant Professor}
    }
    
    \resumeSubheading
      {Beijing University of Posts and Telecommunications, Cyberspace Security}{Beijing, China}     {Bachelor of Engineering}{Aug 2017 - Jul 2021}{
      \resumeItemNoBullet{GPA}{3.66/4}
      \resumeItemNoBullet{Relevant Courses}{Fundamentals of financial security, Assembly language
and reverse engineering, Modern cryptography, OS kernel, Probability and statistics, Fundamentals of web development, Software security, Network security}
      \resumeItemNoBullet{Thesis}{The Design and Implementation of a Tool for Detecting Scam Blockchain Addresses, \href{https://github.com/Kindhearted57/Bachelor_thesis/blob/main/bitcoin_abuse.pdf}{\textcolor{linkblue}{link}}}
      \resumeItemNoBullet{Thesis Abstract}{The upsurge of the blockchain brought by Bitcoin continues to this day. Although there
are a variety of new digital currencies, Bitcoin still attracts a large number of investors by its
currency feature. However, at the same time, the anonymity of Bitcoin has also attracted fraudsters, who are widely using it as an important collection tool.
Bitcoin is currently used in various types of fraud. 
Therefore, investigating and tracking fraudulent crimes becomes imminent. In order to track fradulent behaviors efficiently, finding an effective way to detect fraudulent behaviors in Bitcoin network is of vital importance.
In this thesis, we first collected and labeled a fraudulent
address dataset. The dataset is consisted of Bitcoin addresses identified as fraudulent addresses and
Bitcoin addresses considered to be benign addresses.
We then explore one layer from these two parts of addresses to expand the dataset.
With the dataset available, we firstly analyze some basic features of
the dataset,
and then we explore the validity of using classification approach to detect fraudulent addresses.
Specifically, we firstly use traditional machine learning
algorithms for classification, then we explore using graph neural network on this task. We use both GCN graph neural network and GAT graph attention network to classify the data. Besides the dataset we collected, we also use the public Elliptic data set for comparison. Our results show that the
classifier trained on this dataset has an accuracy higher than 90\% on
the test set.}
    }
    
    
  \resumeSubHeadingListEnd

\shorterSection{\textcolor{titleblue}{Work Experience}}
  \resumeSubHeadingListStart
    \resumeSubheading
      {Huawei Technologies Canada}{Markham, Canada} {Research Associate}{Feb 2022 - now}{
\item{Working with the Formal Method team in formal verification of softwares.}
    }
  \resumeSubHeadingListEnd

% %-----------ACADEMIC PROJECTS AND INTERNSHIPS-----------------
% \shorterSection{Academic Projects and internships}
%   \resumeSubHeadingListStart
%   \small
%     \item{
%      \textbf{Languages}{: Python, C++, SQL, Java, Swift}
%      \hfill
%      \textbf{Technologies}{: GCP, AWS, GitHub, GitLab, Docker}
%     }
%     \vspace{-5pt}
%     \item{
%      \textbf{Libraries}{: TensorFlow, PyTorch, Keras, Scikit-Learn, Numpy, Pandas, Spark, Jupyter, OpenCV, PIL, OpenCL, OpenGL, CUDA}
%     }
% \resumeSubHeadingListEnd

\shorterSection{\textcolor{titleblue}{Publications}}
  \resumeSubHeadingListStart

        \item{Finding Specification Blind Spots via Fuzz Testing, \textbf{Ru Ji}, Meng Xu, \textbf{IEEE Symposium on
Security and Privacy (S\&P) 2023}, \href{https://ieeexplore.ieee.org/document/10179438}{\textcolor{linkblue}{paper}}, \href{https://github.com/watssec/FAST}{\textcolor{linkblue}{code}}, \href{https://www.youtube.com/watch?v=VVjYYDcTbPc&ab_channel=IEEESymposiumonSecurityandPrivacy}{\textcolor{linkblue}{talk}}}
	
	In this paper we proposed a framework named FAST which can examine specification incompleteness by firstly locates specfication gaps and then leverages the test suites to infer whether the gap is introduced by intention or by mistake. FAST is applied to s2n-tls, Diem Payment Network and helped to confirm 13 and 21 blind spots in their specifications respectively.
	
	\textbf{*Real world findings} (Github \href{https://github.com/watssec/FAST}{\textcolor{linkblue}{repo}}): Diem Payment Network - Pull request \href{https://github.com/diem/diem/pull/10152}{\textcolor{linkblue}{1}} \href{https://github.com/diem/diem/pull/10176}{\textcolor{linkblue}{2}} \href{https://github.com/diem/diem/pull/10178}{\textcolor{linkblue}{3}}, s2n-tls - \href{https://mesquite-train-690.notion.site/Missing-specs-in-s2n-tls-90e3e6221e8b42ce84c788491cdc2a3f} {\textcolor{linkblue}{report}}
	
	\textbf{Key skills}: Mutation testing, Fuzzing, LLVM, Smart contracts, Formal verification, C++, Move, Rust
	  
        \vspace{-5pt}
        
        \item{DEPOSafe: Demystifying the Fake Deposit Vulnerability in Ethereum Smart Contracts, \textbf{Ru Ji}, Ningyu He, Lei Wu, Haoyu Wang, Guangdong Bai and Yao Guo, \textbf{International Conference on Engineering of Complex Computer Systems (ICECCS), 2020}, \href{https://ieeexplore.ieee.org/abstract/document/9376204}{\textcolor{linkblue}{paper}}}
        
        Fake deposit is one of the most representative attacks related to exchanges and tokens in blockchain ecosystem, causing large financial losses. We take the first step to demystify the fake deposit vulnerability. We summarize essential patterns and propose an automated tool to detect and verify fake deposit vulnerability in ERC-20 smart contracts. DEPOSafe is applied to 176,000 ERC-20 smart contracts, and identified over 7,000 vulnerable contracts. Our findings demonstrate the urgency to identify and prevent the fake deposit vulnerability.
        
        \textbf{Key skills:} Python, Smart contracts analysis, bash
        
        \vspace{-5pt}
        \item{Characterizing Cryptocurrency Exchange Scams, Pengcheng Xia, Haoyu Wang, Bowen Zhang, \textbf{Ru Ji}, Bingyu Gao, Lei Wu, Xiapu Luo and Guoai Xu, \textbf{Computers \& Security, 2020}, \href{https://www.sciencedirect.com/science/article/abs/pii/S0167404820302662}{\textcolor{linkblue}{paper}}}
        
        Cryptocurrency exchanges attract the attention of attackers, a number of scam attacks were allegedly targeting cryptocurrency exchanges. In this paper we make the first effort to identify and characterize cryptocurrency exchange scams. We identified 94 scam domain families and 30 fake app families.
        
        \textbf{Key skills:} Data collection and analysis, Web crawler, Cluster analysis
        
  \resumeSubHeadingListEnd
  \begin{comment}
%-----------EXPERIENCE-----------------
\shorterSection{\textcolor{titleblue}{Selected Projects}}
  \resumeSubHeadingListStart
    \resumeSubheading
      {Efficient external BST on GPU and OpenMP (a team of two)}{Waterloo, Canada}
      {CS 798 Multicore Programming}{May 2023}
      % what why how
      
       \labelitemii{Generic CPU data structures supported by C++ are not inherently compatible with libraries supporting parallel running.}
       
      \labelitemii{Designed and optimized external binary search tree on OpenMP and GPU.}

  
      \if 0
       \resumeSubheading
      {Performance Analysis of Garbage Collection}{Waterloo, Canada}
      {CS 842 Programming Language}{Dec 2022}
      
          I created baseline, and distributed testing data, performed performance analysis on the garbage collector of Go, C, Python respectively, and provided the results.
      \fi
      \resumeSubheading
      {Blockchain Ledger Condense Algorithm}{Beijing, China}
      {National Cryptography Competition}{Oct 2019}
      
      \labelitemii{Rubbish information on the blockchain ledger occupies the space on the ledger.}
      
      \labelitemii{Used chameleon algorithm to achieve editable blockchain ledger. }
      
      \labelitemii{Implemented verifiable secret sharing to ensure that only certified individuals have the ability to vote for a deletion.}
     
      \resumeSubheading
      {Secure Online Drive (a team of five)}{Beijing, China}
      {}{Aug 2019}
      
     \labelitemii{Created a secure online drive based on C/S architecture.}
     
     \labelitemii{Implemented SSL to ensure the security of data transfer.}
     
     \labelitemii{Implemented AES to ensure the safety of data stored in the server.} 

  \resumeSubHeadingListEnd

\shorterSection{\textcolor{titleblue}{Research Experience}}
  \resumeSubHeadingListStart
    \item{\textbf{Research Assistant in \href{https://howiepku.github.io/}{\textcolor{linkblue}{Security PRIDE}} } \hfill  Beijing University of Posts and Telecommunications, 2019.8 - 2021.6}
   
    \vspace{-5pt}

    \item{\textbf{Research Assistant in \href{https://gossip.team/}{\textcolor{linkblue}{G.O.S.S.I.P}}} \hfill  Shanghai Jiaotong University, 2021.6 - 2021.9}
        
   
    \vspace{-5pt}
    
    \item{\textbf{Research Assistant in \href{https://crysp.uwaterloo.ca/}{\textcolor{linkblue}{CrySP}}} \hfill  University of Waterloo, 2022.1 - 2023.1}
    
    
  \resumeSubHeadingListEnd
  \end{comment}
%-----------PROJECTS/SKILLS-----------------
\if 0
\shorterSection{\textcolor{titleblue}{Teaching Experience}}
  \resumeSubHeadingListStart
    \item{\textbf{Instructor Apprentice, CS245, Logic and Computation} \hfill  University of Waterloo (Fall 2023)}
    
    \vspace{-5pt}

    \item{\textbf{Teaching Assistant, CS458, Computer Security and Privacy} \hfill  University of Waterloo (Spring 2023, Fall 2022)}
        
   
    \vspace{-5pt}
    
    \item{\textbf{Teaching Assistant, CS245, Logic and Computation} \hfill  University of Waterloo (Spring 2022, Winter 2023)}
    

    \vspace{-5pt}

    \item{\textbf{Teaching Assistant, CS116, Introduction to Computer Science 2} \hfill University of Waterloo (Winter 2022)}
    
  \resumeSubHeadingListEnd
\fi
\shorterSection{\textcolor{titleblue}{Awards}}
  \resumeSubHeadingListStart
    \item{\textbf{IEEE Security \& Privacy Student Travel Grant} \hfill San Francisco, US, May 2023}
 
      \vspace{-5pt}
 
      \item{\textbf{Graduate Research Studentship} \hfill University of Waterloo, Jan 2022 - Jan 2024}
 	
      \vspace{-5pt}
      
      \item{\textbf{International Master's Award of Excellence} \hfill University of Waterloo, Jan 2022 - Aug 2023}
      
      \vspace{-5pt}
      
      \item{\textbf{Beijing Outstanding Graduate} \hfill Beijing, China, Jun 2021}
      \if 0
      \vspace{-5pt}
      
      \item{\textbf{2nd Scholarship} \hfill Beijing University of Posts and Telecommunications, China, 2018/2019/2020}
      
       \vspace{-5pt}
       
      \item{\textbf{Outstanding Student Leader} \hfill Beijing University of Posts and Telecommunications, China, 2018/2019/2020}
        
       \vspace{-5pt}
       
      \item{\textbf{First Prize in National Cryptography Competition} \hfill Beijing, China, Nov 2019}
             
      \vspace{-5pt}
      
      \item{\textbf{Second Prize in the National Mathematical Modeling Competition} \hfill Beijing, China, Aug 2019}
      
      \vspace{-5pt}
           
      \item{\textbf{First Prize in the 3rd National English Forum on Electric Power and Humanities} \hfill Beijing, China, May 2019}   \
            
       \vspace{-5pt}     
         
      \item{\textbf{First Prize in National English Competition for University Students} \hfill Beijing, China, May 2018}
       \fi

  \resumeSubHeadingListEnd

%-----------Addtional Experience & Achievements-----------------
\shorterSection{\textcolor{titleblue}{Skills}}
  \resumeSubHeadingListStart

     
      \resumeItemNoBullet{Programming Languages}{Rust, Python, Move, Solidity, C/C++ }
      \resumeItemNoBullet{Techniques}{LLVM, Linux, Pytorch}
      \resumeItemNoBullet{Experienced Research Areas}{System Security, Blockchain Security, Fuzz Testing, Compiler Optimization, Programming Language Security}

     \resumeSubHeadingListEnd

%-------------------------------------------
\end{document}
